\begin{appendices}
\section{闪电网络技术的拓展}
自从闪电网络的白皮书公开之后,社区对其技术做了很多更加深入的研究,不断的优化去信任的实时清算协议。
下面我们介绍三个相关的进展。

\subsection{Sprites风格的 HTLC}
\href{https://arxiv.org/abs/1702.05812}{Sprites and State Channels: Payment Networks that Go Faster than Lightning} 是2017年发表的一篇论文,它提出了一种新的 HTLC 承诺方案。这种方案被称为 Sprites-风格的 HTLC,对应的,闪电网络协议中 HTLC 被称为闪电-风格的HTLC。这种新的 HTLC 在2个方面对闪电网络协议提出了改进。

\begin{itemize}
    \item 支持支付通道部分存取款

    在闪电网络协议中,用户一旦从虚拟银行中取款,其中的资产被全部结算,支付通道随即关闭。
    Sprites-风格 HTLC 支持部分取款。在不关闭虚拟银行的情况下,支付双方向可以向虚拟银行追加资金、或者部分提现。
    这样提高了支付通道的利用率,也节约了重新开启支付通道的成本。

    \item 抵押资产优化

    在闪电网络的协议中,HTLC 时间锁的大小和支付路径的长度有关系。
    假设支付路径的长度为L,如下图所示,按照从接收方到支付方的顺序,时间锁的大小分别为:T,2T ... L*T。
    在最糟糕的情况下,资产的锁定时间随着支付路径的增加而线性增长。
    Sprites-风格的HTLC 做了优化,令资产的锁定时间与路径的长度无关。
\end{itemize}

\subsection{Perun: 虚拟支付通道}
在论文 \href{https://eprint.iacr.org/2017/635.pdf}{Perun: Virtual Payment Hubs over Cryptocurrencies} 中,又提出了一种新的支付通道链接技术,称之为“虚拟支付通道”,进一步改进了 HTLC。

这种技术扩展了虚拟银行智能合约,为支付双方提供了额外的功能。
举例来说,假设 Alice 和 Carol 之间没有支付通道,但是他们分别和 Bob 有支付通道连接。
在闪电网络的协议中,Bob 作为中间人要分别和 Alice 和 Carol 进行支付。
但是在虚拟支付通道的协议中,Bob 并不需要确认双方的交易,甚至于 Bob 临时的离线也没有影响。
虚拟通道技术可以进一步降低交易的延时和费用,同时提高了支付系统的可用性。
但是要指出的是,论文中只讨论了一个中间节点的情况,如果支付路径包含多个中间节点,依然是一个开放的研究课题。

\subsection{广义状态通道 Generalized State Channels}
支付通道的概念可以推广成为状态通道。在一个状态通道中,Alice 和 Bob 可以完成相对于支付更复杂的链下智能合约功能。
比如说,在线游戏、资产交易等。在闪电网络中,Alice 和 Bob 在链下共同管理虚拟银行的债务分配方案。
类似的,在状态通道中,双方在链下共同维护一个智能合约的状态,通过一种二元共识协议,对状态更新达成一致,不需要每次都公布到链上。
任何一方都随时可以公开链下最新的状态并且同步到链上。

论文 \href{https://www.counterfactual.com/statechannels/}{Counterfactual: Generalized State Channels} 系统的提出了广义状态通道的概念,并且为开发者提供了状态通道的开发框架,Dapp开发者使用其 API 就能方便的集成状态通道的技术。
值得注意的是,此论文提出了 \textbf{Counterfactual} 的概念,用于概括哪些行为不用上链,可以在状态通道中管理。
假设一个链上的真实事件为 X,已经被矿工确认,具有不可篡改、不可伪造、不可撤销性。那么对应的链下事实为 \textbf{Counterfactual of X},它满足3个条件:

\begin{enumerate}
    \item 事件 X 还没有在链上被确认,也就是说它还没有被广播到链上。
    \item 状态通道保证任何参与者都可以单方面广播 X,并且无风险的被矿工确认,成为链上事实。
    \item 相关参与者的行为都假设 X 已经在链上发生了,仿佛 X 真的成为不可篡改、不可撤销的事实。
\end{enumerate}

以支付通道为例,一个支付事件 X 可以是 “在虚拟银行中,Alice 的账户减去 10 美元,Bob的账户增加 10 美元”, 那么对应的 \textbf{Counterfactual of X} 可以是一个 RSMC 承诺方案:“Alice 和 Bob 共同签署一个承诺方案:Alice 的账户减去 10 美元,Bob 的账户增加 10 美元”。
所以闪电网络中的承诺方案可以看成是\textbf{Counterfactual of X}的一个特例。

这个概念非常有价值,它不但给出了通用的状态通道的基本概念,而且也指出了链上交易的通道化的本质:构建一个可信的机制,使得智能合约的所有参与方,对于将要发生的链上事实提前达成一致,形成 \textbf{Counterfactual of X}。此事实虽然未发生,但是等同于发生,并且彼此互相信任对方,任何人都无法反悔。

\end{appendices}
