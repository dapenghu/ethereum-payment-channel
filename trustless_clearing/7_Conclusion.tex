\section{总结,现代支付清算结算系统与应用}

在现代金融系统中,有两种并行运作的支付体系:一个是基于所有权交割的现金支付体系,另一个是基于债权清算的银行支付体系。现金支付体系是一种古老的支付模式,具有普适性、门槛低的优势,基本上所有国家和地区都能使用;银行支付体系的优势是:安全、便捷、高效、支持远距离支付;但是需要现代银行系统和金融IT系统的支持,对金融监管、风险控制、网络安全、智能终端等基础设施要求很高,一般只有经济较为发达的国家才能建设。这两种支付体系互相独立、互相关联,互相弥补,共同支撑了金融系统的运行。

以比特币为代表的区块链技术基于互联网技术构建了电子现金支付系统,替代了实物现金,大大降低了现金的存储、携带等管理成本,同时也提高了支付的便捷性、安全性。但是对比于基于债务清算的电子支付系统,由于所有权交割的天然缺陷,交易的成本很高,一直受到扩容问题的困扰,吞吐量低下限制了区块链在金融系统中的普及。

区块链技术经过十年的发展,不断探索、不断创新,以闪电网络为代表的债务清算技术逐步成熟。它使用虚拟银行智能合约代替传统的银行机构托管用户的资产,在链下构建了均衡的博弈规则对资产进行清算。这种新的清算方式一方面保留了无需信任背书和外部金融监管的优势,同时也提升了支付系统的容量和隐私性。**从某种意义上说,比特币通过现代IT技术还原了最原始的价值转移方式,而闪电网络又把我们拉回到了现代。**

这两种区块链技术互相关联,互相弥补,共同为金融提供新的技术解决方案。相对于传统的支付系统,区块链技术有一下优势:

\begin{itemize}
    \item 低摩擦
    
    区块链有内在的信用机制可以防范支付过程中的对手风险、交易风险,大大节约了对于金融机构的监管、合规成本。这些成本上节约最终会降低支付系统的摩擦,令消费者受益。
    
    \item 轻金融
    
    传统金融机构有体量大、成本高、组织结构复杂等特点,没有为社会所有的人群提供有效的服务。而是倾向于为大机构、大企业、富有群体提供金融服务。相对来讲,使用区块链开展金融业务的机构更加灵活、更加轻量级,能有效、全方位地为社会所有阶层和群体提供服务。
    
    \item 易普及
    
    区块链基于互联网技术,而且对于可信金融机构的依赖程度低。对于欠发达地区来讲,构建电子支付系统的门槛大大降低了。使金融负能够惠及全球更多人口。
   
\end{itemize}

在未来几年,区块链作为基础设施逐渐应用于各种金融服务,逐渐表现出强大的技术创新性,市场和消费者肯定受益颇丰。

