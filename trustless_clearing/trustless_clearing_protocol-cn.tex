%!TEX program = xelatex

\documentclass[lang=cn]{elegantpaper}

\title{ElegantPaper: 一个优美的 \LaTeX{} 工作论文模板}

\author{\href{https://github.com/dapenghu}{胡 大 鹏}\thanks{}}

\institute{OK区块链工程院}

\version{0.06}

\date{\today}

\begin{document}

\maketitle

\begin{abstract}

\noindent 比特币 Bitcoin, Ethereum and other blockchain-based cryptocurrencies, as deployed today, cannot scale for wide-spread use. A leading approach for cryptocurrency scaling is a smart contract mechanism called a payment channel which enables two mutually distrustful parties to transact efficiently (and only requires a single transaction in the blockchain to set-up). 
Payment channels can be linked together to form a payment network, such that payments between any two parties can (usually) be routed through the network along a path that connects them. Crucially, both parties can transact without trusting hops along the route.
In this paper, we propose a novel variant of payment channels, called Sprites, that reduces the worst-case “collateral cost” that each hop along the route may incur. The benefits of Sprites are two-fold. 
1) In Lightning Network, a payment across a path of ` channels requires locking up collateral for ⇥(` ) time, where   is the time to commit an on-chain transaction. Sprites reduces this cost to ⇥(` +  ). 
2) Unlike prior work, Sprites supports partial withdrawals and deposits, during which the channel can continue to operate without interruption.
In evaluating Sprites we make several additional contributions. First, our simulation-based security model is the first formalism to model timing guarantees in payment channels. Our construc- tion is also modular, making use of a generic abstraction from folklore, called the “state channel,” which we are the first to formalize. We also provide a simulation framework for payment network protocols, which we use to confirm that the Sprites construction mitigates against throughput-reducing attacks.

\end{abstract}

\section{示例}

%----------------------------------------------------------------------------------------
%	BIBLIOGRAPHY
%----------------------------------------------------------------------------------------


%----------------------------------------------------------------------------------------
%	REFERENCE LIST
%----------------------------------------------------------------------------------------
\newpage

%----------------------------------------------------------------------------------------
%	BIBLIOGRAPHY
%----------------------------------------------------------------------------------------


\bibliographystyle{unsrt}
\bibliography{sample.bib}

%\begin{thebibliography}{99} % Bibliography - this is intentionally simple in this template
%
%  \bibitem{btc_wp} Satoshi Nakamoto.
%  \newblock \textit{Bitcoin: A peer-to-peer electronic cash system (2008).}
%  \newblock \url{https://bitcoin.org/bitcoin.pdf}.
%
%  \bibitem{eth_wp} Vitalik Buterin.
%  \newblock \textit{Ethereum: a next generation smart contract and decentralized application platform (2013).}
%  \newblock \url{https://github.com/ethereum/wiki/wiki/White-Paper}.
%
%  \bibitem{xlm_wp} David Mazi\`eres.
%  \newblock \textit{The Stellar Consensus Protocol: A Federated Model for Internet-level Consensus (2016).}
%  \newblock \url{https://www.stellar.org/papers/stellar-consensus-protocol.pdf}.
%
%  \bibitem{zeppelin} \textit{OpenZeppelin.}
%  \newblock \url{https://openzeppelin.org/}.
%
%\end{thebibliography}
 % Specifies the document structure and loads requires packages

\end{document}
