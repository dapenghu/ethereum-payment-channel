\section{介绍}

闪电网络(Lightning Network)是 Joseph Poon 和 Thaddeus Dryja 在2015年合著的白皮书中提出的。它在比特币社区中产生了很大反响,在众多关于比特币的论文和白皮书中,被认为是第二重要的,其价值仅次于中本聪的创世论文。

由于闪电网络依赖于隔离验证,在2017年比特币隔离验证升级之前,一直停留在概念和内部开发阶段。2018年3月,Lightning Labs 开发并推出了第一个测试版,之后 ACINQ 和 Blockstream 两家公司也相继推出了不同的实现。从此以后闪电网络的发展就步入正轨。根据 \href{https://1ml.com/statistics}{} 的数据,闪电网络目前有 6,761 个节点,30,622 个支付通道,支付通道总计有 729.89 BTC (约281万美金)。说明闪电网络在过去的一年中取得了显著增长。

闪电网络的愿景是解决比特币网络的扩容问题。众所周知,比特币的初衷是实现一个端到端的电子现金系统,为全世界提供一个去信任的、7x24小时服务的电子支付网络。但是比特币的性能却远远达不到要求。按照平均每个交易300字节计算,比特币的平均吞吐量是 5.6 TPS。然而 \href{https://www.visa.com/blogarchives/us/2013/10/10/stress-test-prepares-visanet-for-the-most-wonderful-time-of-the-year/index.html}{Visa} 的峰值吞吐量可以达到 47000 TPS。如果对标这个吞吐量,比特币的区块大小要扩张到 8GB 左右,每年要新增 400 TB的区块数据。这显然是不现实的。

除了闪电网络,比特币社区同时也提出了众多的扩容解决方案,比如大区块、DPoS、DAG、分片等。这些方案试图修改比特币协议本身,例如调整配置参数、优化数据结构、修改共识算法、账本分区处理、优化网络资源管理等等。但是效果都不好,在付出了高昂的代价(增加存储量、增加网络流量、增加逻辑复杂度、弱化去中心化)之后,却只是获得了非常有限的性能提升,和 Visa 相比依然还有几个数量级的差距。

唯有闪电网络脑洞大开、另辟蹊径,使用了新的价值转移范式。在比特币的智能合约基础上,提出了基于支付通道的去信任的清算协议,构建了二层清算系统。彻底摆脱了“去中心化-成本-性能”的三元悖论约束,将系统的并发量上限提升到了几十万TPS级别,而且可以做到实时确认,达到了类似于支付宝、微信支付的使用体验。而且难能可贵的是,它对比特币网络本身几乎没有带来任何负面影响(隔离验证对于比特币的负面影响很小)。

闪电网络并没有使用类似于零知识证明那样的高难度技术,但是它的巧妙设计依然令白皮书晦涩难读。市面上也缺乏简单易读而且讲解透彻的科普文章,对于广大金融科技与区块链爱好者和投资者来讲,有很高的学习门槛。所以闪电网络技术的价值长期被误解、被低估。本文重新梳理了闪电网络的思想,用通俗易懂的文字,为大家介绍闪电网络的技术原理,总结技术优势和劣势,分析它的适用的场景,最终阐述它在现代电子支付系统中的潜在应用价值。希望能帮助广大读者更深入的认知闪电网络。

此文的章节结构如下:

\begin{itemize}
    \item 第二节介绍两种不同的价值转移范式,分析比特币扩容困境的根本原因、以及闪电网络的思路。
    \item 第三节介绍闪电网络的基本概念和原理
    \item 第四节结合Solidity智能合约的源码介绍技术细节。
    \item 第五节分析闪电网络技术的优点和限制。
    \item 第六节介绍支付通道技术的几种技术进展和优化
    \item 第七节总结去信任实时清算协议对于区块链和金融科技的价值和意义。
\end{itemize}
