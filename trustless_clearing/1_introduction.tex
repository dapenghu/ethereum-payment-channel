\section{介绍}

闪电网络(Lightning Network)是 Joseph Poon 和 Thaddeus Dryja 在2015年合著的白皮书中提出的。它在比特币社区中产生了很大反响,在众多关于比特币的论文和白皮书中,被认为是第二重要的,其价值仅次于中本聪的创世论文。

由于闪电网络依赖于隔离验证,一直停留在概念和内部开发阶段。从2017年比特币隔离验证分叉之后步入正常的发展轨道,2018年3月,Lightning Labs 开发并推出了第一个测试版,之后 ACINQ 和 Blockstream 两家公司也相继推出了不同的实现。根据统计网站  \href{https://1ml.com/statistics}{1ml} 的数据,闪电网络目前有 7,634 个节点,39,409 个支付通道,支付通道总计有 1,050.19 BTC (约413万美金)。说明闪电网络在过去的一年中取得了显著增长。

闪电网络的愿景是解决比特币网络的扩容问题。众所周知,比特币的初衷是实现一个端到端的电子现金系统,为全世界提供一个去信任的、7x24小时服务的电子支付网络。但是比特币的性能却远远达不到要求。按照平均每个交易300字节计算,比特币的平均吞吐量是 5.6 TPS。然而 \href{https://www.visa.com/blogarchives/us/2013/10/10/stress-test-prepares-visanet-for-the-most-wonderful-time-of-the-year/index.html}{Visa} 的峰值吞吐量可以达到 47,000 TPS。如果对标这个吞吐量,比特币的区块大小要扩张到 8GB 左右,每年要新增 400 TB的区块数据。这显然是不现实的。

除了闪电网络,比特币社区同时也提出了众多的扩容解决方案,比如大区块、DPoS、DAG、分片、侧链跨链等。这些方案在比特币的分布式账本技术基础上做了优化,例如调整配置参数、优化数据结构、修改共识算法、账本分区处理、优化网络资源管理等等。但是效果都不好,在付出了高昂的代价(增加存储量、增加网络流量、增加逻辑复杂度、弱化去中心化)之后,却只是获得了非常有限的性能提升,和 Visa 相比依然还有几个数量级的差距。

唯有闪电网络脑洞大开、另辟蹊径,使用了与比特币不同的价值转移范式。它提出了基于支付通道的去信任的清算协议,在比特币网络上构建了二层清算系统。彻底摆脱了分布式账本的“去中心化-成本-性能”的三元悖论约束。不但将系统的并发量上限提升到了几十万TPS级别,而且可以做到实时确认,达到了类似于支付宝、微信支付的使用体验。难能可贵的是,它对比特币网络本身的负面影响非常小(隔离验证对于比特币的负面影响很小)。

闪电网络并没有使用类似于零知识证明那样的高难度技术,但是它的巧妙设计依然令白皮书晦涩难读。
市面上也缺乏简单易读而且讲解透彻的科普文章,对于广大金融科技与区块链爱好者和投资者来讲,有很高的学习门槛。
所以闪电网络技术的价值长期被误解、被低估。
本文使用 Solidity 语言重新实现了闪电网络,绕过了比特币智能合约语言的复杂性,重新梳理了它的基本思想。
用通俗易懂的文字,为大家介绍闪电网络的技术原理,总结技术优势和劣势,分析它的适用的场景,最终阐述它在现代电子支付系统中的潜在应用价值。
希望能帮助广大读者更深入的认知闪电网络。

此文的章节结构如下:

\begin{itemize}
    \item 第二节从数字货币两个基本问题切入,介绍比特币的技术思路,分析比特币扩容困境的根本原因。
          然后对比两种不同的价值转移范式,介绍为什么闪电网络能够跳出三元不可能原理的限制,从而大大提升系统的性能。
    \item 第三节把闪电网络的基本概念和原理抽象出来,提出虚拟银行、共同承诺、支付通道等新概念,并且提出去信任的实时清算协议框架。
    \item 第四节分析闪电网络技术的优点和限制。
    \item 第五节总结去信任实时清算协议对于区块链和金融科技的价值和意义。
    \item 在附录A中,介绍虚拟银行智能合约的 Solidity 源码,详细阐述如何与 RSMC、HTLC 承诺方案配合的技术细节。
    \item 在附录B中,介绍支付通道技术的几种技术进展和优化,更加强大的 Solidity 智能合约语言,闪电网络的技术可以继续拓展。
\end{itemize}
