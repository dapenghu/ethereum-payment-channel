\section{技术优势和劣势分析}
下面我们逐一分析这种新技术的优势和劣势。

\subsection{技术优势}
根据闪电网络技术的分析,支付通道技术通过链下的共同承诺方案完成支付,把参与共识的范围大大缩小了,会带来很多优势:

\begin{itemize}
    \item  \textbf{交易费用低}
        交易过程中,需要借用中间节点提供的资金流动性,所以需要支付一定的费用。
        但是由于中间节点是非垄断性的、而且是去信任的,没有监管、合规等成本,费用非常的低。
        相对于 Visa 的 $2~3\%$ 的费率,闪电网络收取的费用几乎可以忽略不计。
        
    \item \textbf{交易实时确认}
        支付的过程中无需大量节点参与共识,只需要参与者之间交换共同承诺方案的参数和签名信息,可以实时的完成交易的确认。具体的来讲,交易确认的时间包含:
        \begin{itemize}
            \item 支付通道路由选择。如果支付双方没有直接联系的支付通道,需要通过路由算法找到合适的支付路径。
            \item 支付通道数据交互。每一次承诺方案编号加一,需要2次数据交换:一次是交换新的承诺方案,一次交换旧承诺方案的撤销锁私钥。如果仅仅涉及到 RSMC 承诺方案,每次支付编号只加一次;但是一般情况下是 HTLC 和 RSMC 互相配合使用,承诺方案编号增加2,所以一共需要4次数据交互。
            
        \end{itemize}
        所以,连个节点之间一次交互的时间一般在几十至几百毫秒之间,所以总的确认时间大约在几百毫秒至几秒之间。

      \item \textbf{高并发性}

        不同的支付路径是互相独立的,可以并行执行,相对于链上的交易,系统的并发性被大幅提升。
        我们大致的估算一下闪电网络并发性的理论上限。
        对于正常的一个支付通道来讲,只有建立和关闭发生在链上,一共有4个交易(两个充值、两个取款,闪电网络不需要部署虚拟银行智能合约)。
        那么底层区块链的吞吐量限制了支付通道的数量,假设平均每个通道的生命周期是 N 天,对应的通道数量的计算公式为:

        \begin{equation}
                    \text{支付通道数量上限} = \text{比特币TPS} \times 3600 \times 24 \times N \div 4
        \end{equation}
        
        根据闪电网络统计网站 \href{https://1ml.com/statistics}{1ML}的数据,支付通道的平均生命周期 N 为 54.9 天。
        假设比特币的吞吐量为 3.33 TPS,那么对应的支付通道上限为: 3,952,800。

        根据6度空间理论,任何两个陌生人之间的间隔不会超过六个人,也就是说,支付路径的最大长度一般不会超过6。
        那么根据下面公式可以算出,闪电网络可以同时支持 $658,800$ 个支付的并发执行。
        \begin{equation}
                    \text{闪电网络并发量} = \frac{\text{支付通道数量上限}}{\text{平均每笔交易占用通道数}} = \frac{3,952,800}{6} = 658,800
        \end{equation}

      \item \textbf{数据存储小}

        交易在链下完成的另一个好处是节约数据存储成本。
        交易数据只需要在参与方之间传播,历史承诺方案可以立刻丢弃,只需要保存对应的撤销锁私钥列表。
        相对于链上交易来讲,大大降低了数据存储的需求。

      \item \textbf{隐私性}

        除了支付通道的开启和关闭,大部分交易都发生在区块链之外,没有广播到链上,只有通道的参与者了解交易的信息。
        因此所有微支付几乎都无法追踪。对于支付具有天然的隐私保护作用。

\end{itemize}

\subsection{缺点}

另一方面,支付通道的支付方式也有限制条件和缺陷,只有在合适的条件下才能充分发挥它的价值。

\begin{itemize}
    \item \textbf{高频往来交易}

        支付通道的建立是有成本的,开启和关闭支付通道需要至少4笔交易。
        如果在支付通道的存续期间只有少量的支付,那么支付通道的意义就不大了,不如直接在链上交易。
        最适合支付通道的应用场景是双方需要频繁的往来交易,比如说银行间支付清算业务。
        两个银行之间建立支付通道,为其用户之间的往来支付提供清算服务,可以最大的提高支付通道的利用率。

    \item \textbf{资金锁定}
        支付通道的中间节点需要提供一定的资产,这部分资产长期处于锁定状态,牺牲了一部分资金的流动性。
    
    \item \textbf{双方必须在线}
        虽然承诺方案的协商发生在链下,但是虚拟银行的双方都需要监控链上交易,及时发现对方提交的承诺方案,才能避免自己的损失。
        所以支付通道的双方依然需要在线监听链上的交易。
        这个要求对B端用户没有太大影响,但是对于C端用户来讲,是一个比较强的要求。

\end{itemize}

